% Options for packages loaded elsewhere
\PassOptionsToPackage{unicode}{hyperref}
\PassOptionsToPackage{hyphens}{url}
%
\documentclass[
  x11names]{article}
\usepackage{amsmath,amssymb}
\usepackage{iftex}
\ifPDFTeX
  \usepackage[T1]{fontenc}
  \usepackage[utf8]{inputenc}
  \usepackage{textcomp} % provide euro and other symbols
\else % if luatex or xetex
  \usepackage{unicode-math} % this also loads fontspec
  \defaultfontfeatures{Scale=MatchLowercase}
  \defaultfontfeatures[\rmfamily]{Ligatures=TeX,Scale=1}
\fi
\usepackage{lmodern}
\ifPDFTeX\else
  % xetex/luatex font selection
\fi
% Use upquote if available, for straight quotes in verbatim environments
\IfFileExists{upquote.sty}{\usepackage{upquote}}{}
\IfFileExists{microtype.sty}{% use microtype if available
  \usepackage[]{microtype}
  \UseMicrotypeSet[protrusion]{basicmath} % disable protrusion for tt fonts
}{}
\makeatletter
\@ifundefined{KOMAClassName}{% if non-KOMA class
  \IfFileExists{parskip.sty}{%
    \usepackage{parskip}
  }{% else
    \setlength{\parindent}{0pt}
    \setlength{\parskip}{6pt plus 2pt minus 1pt}}
}{% if KOMA class
  \KOMAoptions{parskip=half}}
\makeatother
\usepackage{xcolor}
\usepackage[margin=1in]{geometry}
\usepackage{graphicx}
\makeatletter
\def\maxwidth{\ifdim\Gin@nat@width>\linewidth\linewidth\else\Gin@nat@width\fi}
\def\maxheight{\ifdim\Gin@nat@height>\textheight\textheight\else\Gin@nat@height\fi}
\makeatother
% Scale images if necessary, so that they will not overflow the page
% margins by default, and it is still possible to overwrite the defaults
% using explicit options in \includegraphics[width, height, ...]{}
\setkeys{Gin}{width=\maxwidth,height=\maxheight,keepaspectratio}
% Set default figure placement to htbp
\makeatletter
\def\fps@figure{htbp}
\makeatother
\setlength{\emergencystretch}{3em} % prevent overfull lines
\providecommand{\tightlist}{%
  \setlength{\itemsep}{0pt}\setlength{\parskip}{0pt}}
\setcounter{secnumdepth}{-\maxdimen} % remove section numbering
\usepackage{bbm}
\usepackage{array}
\usepackage{tabulary}
\usepackage{hanging}
\usepackage{hhline}
\usepackage{multirow}
\usepackage{fancyhdr}
\pagestyle{fancy}
\usepackage{colortbl}
\AtBeginDocument{\let\maketitle\relax}
\usepackage{makecell}
\usepackage{hyperref}
\usepackage{graphicx}
\ifLuaTeX
  \usepackage{selnolig}  % disable illegal ligatures
\fi
\usepackage{bookmark}
\IfFileExists{xurl.sty}{\usepackage{xurl}}{} % add URL line breaks if available
\urlstyle{same}
\hypersetup{
  pdftitle={PS4168 Economic Psychology},
  pdfauthor={Dr Cillian McHugh},
  hidelinks,
  pdfcreator={LaTeX via pandoc}}

\title{PS4168 Economic Psychology}
\usepackage{etoolbox}
\makeatletter
\providecommand{\subtitle}[1]{% add subtitle to \maketitle
  \apptocmd{\@title}{\par {\large #1 \par}}{}{}
}
\makeatother
\subtitle{Module Handbook}
\author{Dr Cillian McHugh}
\date{Autumn Semester 24/25}

\begin{document}
\maketitle

\fancyhead{}
\fancyfoot{}
\fancyhead[LO,RE]{{PS4168 Economic Psychology | Module Handbook}}
\fancyhead[LE,RO]{\thepage} 
\fancyfoot[LO,RE]{Autumn 2024}
\fancyfoot[LE,RO]{Dr Cillian McHugh}

~

~

~

\centering\includegraphics[width=2.08333in,height=\textheight]{"logos/department_logo.PNG"}

\raggedright

~

~

~

\renewcommand{\arraystretch}{1.5}

\begin{center}


\bigskip


\bigskip

\begin{huge} PS4168 Economic Psychology
\end{huge}

\begin{large} Module Handbook

\textit{Dr Cillian McHugh}

\textit{Autumn Semester 2024}
\end{large}
\end{center}

\pagebreak

\section{PS4168 Economic Psychology - Module
Handbook}\label{ps4168-economic-psychology---module-handbook}

\begin{minipage}{.65\textwidth}
\begin{tabular}{p{4cm} l}
\textbf{Module Leader} & Dr Cillian McHugh\\
\textbf{Email} & cillian.mchugh@ul.ie \\
\textbf{Office} & KBG-22 \\
\textbf{Student Hours} & Wednesdays, 9-11  \\
\textbf{Lecture location} & SR2030\\
\textbf{Day and Time} & Fridays, 9-11 \\
\end{tabular}
\end{minipage}
\hspace{6pt}
\begin{minipage}{.15\textwidth}
\includegraphics[width=75px]{images/cillian_face_2022e.jpg}\bigskip
\end{minipage}

\bigskip

Office hours can be online or in-person please
\color{blue}\href{https://outlook.office.com/bookwithme/user/f570b7289a5343259c60c8a5d26ce510@ul.ie/meetingtype/w1pxpwCL8UG9pt_0cg2fUA2?anonymous&ep=mlink}{book
an appointment here}\color{black}.

Collaborative note taking
\color{blue}\href{https://docs.google.com/document/d/1Yx5EbhSX0uEHzr964dLh7Xx30ZzthbCqV3T32laTl9E/edit?usp=sharing}{here}\color{black}

\subsection{Module Overview}\label{module-overview}

Economic psychology is the study of the psychological processes that
underlie economic behaviour and decision making. It is influenced by the
psychology of decision making and behavioural economics. This module
provides an overview of key concepts in economic psychology and
behavioural economics. A range of historical and contemporary theories
of decision making will be covered, along with various influences on
decision making, preference, and behaviour.

A primary focus of this module is the application of insights from
behavioural studies to theories of decision making, and the application
of these theories in predicting patterns of economic behaviour. The
ethical implications of these insights will also be explored. The
primary reading material will be taken from Ranyard (2018), Altman
(2017), and Camerer, Loewenstein, \& Rabin (2003). Additional reading
may be found in Frantz, Chen, Dopfer, Heukelom, \& Mousavi (2016) and
Cartwright (2014).

\subsection{Learning Outcomes}\label{learning-outcomes}

On completion of PS4168 students should be able to:

\begin{enumerate}
\def\labelenumi{\arabic{enumi}.}
\item
  Show evidence of an understanding of some of the key concepts and
  theories in Economic Psychology.
\item
  Apply findings of psychological research to everyday life and real
  world phenomena.
\item
  Synthesise their acquired knowledge to explore possible alternative
  solutions to practical problems.
\item
  Articulate and defend an informed opinion on a theoretical position.
\end{enumerate}

\bigskip

\emph{Cognitive (Knowledge, Understanding, Application, Analysis,
Evaluation, Synthesis)}

On successful completion of this module, students should be able to:

\begin{itemize}
\tightlist
\item
  Describe the characteristics of economic behaviour.
\item
  Describe theoretical models used in economic psychology.
\item
  Critically appraise these models from a theoretical perspective with
  reference to empirical evidence.
\item
  Explain major models used in economic psychology and decision making.
\item
  Critically appraise the application of theory to real world settings.
\end{itemize}

\bigskip

\emph{Affective (Attitudes and Values)}

On successful completion of this module, students should:

\begin{itemize}
\tightlist
\item
  Value the contribution that psychological research and theory building
  to the understanding of economic behaviour.
\item
  Value the role of economic psychology at the individual, interpersonal
  and societal level.
\end{itemize}

\bigskip

\emph{Psychomotor (Physical Skills)}

N/A

\subsubsection{Graduate Attributes}\label{graduate-attributes}

The learning outcomes identified above aim to contribute to each of the
UL Graduate Attributes in the following ways:

\emph{Articulate}

\begin{itemize}
\tightlist
\item
  This module makes use of in-class discussion/group-work as well as
  with written assignments.
\end{itemize}

\emph{Agile}

\begin{itemize}
\tightlist
\item
  Real-world examples will be employed for illustrating the novel
  theoretical constructs. Furthermore, the assessments involve students
  identifying their own novel real-world examples.
\end{itemize}

\emph{Courageous}

\begin{itemize}
\tightlist
\item
  Students will be asked to devise creative solutions to real-world
  problems. In-class discussions will include being challenged on, and
  being asked to defend a position.
\end{itemize}

\emph{Curious}

\begin{itemize}
\tightlist
\item
  Self-directed learning is core to doing well in this module. Students
  are encouraged to read beyond the prescribed materials and explore the
  relevant literature to gain deeper understandings of the topics
\end{itemize}

\emph{Responsible}

\begin{itemize}
\tightlist
\item
  The content of the module includes considerations of the ethical
  implications of the application of the concepts covered. For example,
  in addition to exploring the practicalities and applications of
  theoretically informed interventions, the ethical considerations of
  such approaches will also be explored.
\end{itemize}

\subsection{Teaching and Learning
Strategies}\label{teaching-and-learning-strategies}

The module will be taught through lectures. The lectures will have
opportunities for discussion and more in depth exploration of the
material along with relevant activities to help consolidate learning.

\subsection{Assessment}\label{assessment}

The assessment for this module will be a combination of coursework and
an exam. There will be an exam (essay questions) worth 45\%. There will
also be two short tasks, worth 25\% and 30\%. Marking schemes for each
assignment are provided below. It is expected that each piece of course
work will be appropriately referenced and adhere to APA standards. Marks
will not be awarded for this, however significant failure in this regard
may result in a penalty of up to a sub-grade being applied. The
breakdown of marks and important deadlines are outlined in Table 1:

\bigskip

\begin{tabular}{l >{\centering\arraybackslash}m{.5cm} c >{\centering\arraybackslash}m{.5cm} l}
  \multicolumn{5}{l}{\textbf{Table 1:} Important deadlines} \\
  \hline \hline
  \textbf{Type of Assessment}               & & \textbf{\% of Final Grade} & & \textbf{Deadline} \\ \hline
  Task 1                                    & & 25\%                       & & Wednesday 16th October at 4pm \\ \hline
  Task 2                                    & & 30\%                       & & Wednesday 13th November at 4pm \\ \hline
  Exam                                     & & 45\%                       & & Exam Week (1.5 hour) \\ \hline \end{tabular}

\bigskip

\subsection{Assessment and Grades}\label{assessment-and-grades}

\subsubsection{Short Tasks}\label{short-tasks}

The short tasks will focus on individual topics as follows:

\begin{itemize}
\tightlist
\item
  \textbf{Task 1} \emph{(select one)}:

  \begin{itemize}
  \tightlist
  \item
    Heuristics
  \item
    Nudges
  \end{itemize}
\item
  \textbf{Task 2}:

  \begin{itemize}
  \tightlist
  \item
    Game Theory
  \end{itemize}
\end{itemize}

You will have at least 2 weeks to complete each task. The specific
nature of each task will vary depending on the topic. They share a
common marking scheme described in Table 2: (The assignment should be
within the designated word count +/- 10\%)

\bigskip

\begin{tabular}{  >{\arraybackslash}m{10cm} >{\centering\arraybackslash}m{5.5cm} }
\multicolumn{2}{l}{\textbf{Table 2:} Marks allocation for Tasks 1 and 2} \\
\hline \hline
\textbf{Component} & \textbf{Percentage} \\ \hline
Clear demonstration of an in depth understanding of the topic & 30\% \\ \hline
Critical thinking & 30\% \\ \hline
Competence in the identification and application of relevant research methods & 20\% \\ \hline
Originality/Novelty & 20\% \\ \hline
References, Citations, and Formatting & (potential penalty of 1 sub-grade) \\ \hline
\end{tabular}

\bigskip

\subsubsection{Exam}\label{exam}

45\% of the module will be an Essay based exam. You will receive 3 essay
titles and you can pick one.

\bigskip

\begin{tabular}{ >{\arraybackslash}m{9cm} >{\centering\arraybackslash}m{6.5cm} }
\multicolumn{2}{l}{\textbf{Table 3:} Marks allocation for Essay-Exam} \\
\hline \hline
\textbf{Component} & \textbf{Percentage} \\ \hline
Content & 70\% \\ \hline
Structure, Language, and Written Expression & 30\% \\ \hline
References, Citations, and Formatting & (potential penalty of 1 sub-grade) \\ \hline
\end{tabular}

\bigskip

\emph{Content}

The assignment should directly address the question asked and
demonstrate a clear coherent argument throughout. Excessive descriptions
should be avoided if they do not serve to support an argument. Evidence
of critical thinking is essential. Students should demonstrate evidence
of wider reading and understanding of the concepts being discussed.

\emph{Stucture, Language, and Written Expression}

Ideas should be clearly expressed, with appropriate use of scientific
language. There should be a clear introduction, body, and conclusion,
with appropriate use of paragraphs (The use of sub-headings is
optional). The assignment should be within the designated word count
(+/- 10\%).

\emph{References, Citations, and Formatting}

In-text citations and the References section should be in APA style. All
citations should appear in the References section, and all references
should be cited in the text. A References section should be included
(i.e., no bibliography). A cover page should be included, and the
assignment should be double-spaced.

\subsubsection{Departmental Penalty
Policies:}\label{departmental-penalty-policies}

Late Submission of assignment: A subtraction of 5\% per day from the
total grade, up to 5 days, after which a mark of zero will be given. A
soft copy must be uploaded by the deadline.

The Department of Psychology does not offer extensions; if you miss an
assessment deadline, you must submit an M-Form with documentary evidence
supporting your reason. This is considered by a committee at the end of
the semester. Please carefully read the Department of Psychology Student
Handbook; especially the sections relating to Mitigating Circumstances.
Please note that these rules apply to all students on the module
including those who register late and those from other
faculties/disciplines.

Excuses: If you have a valid excuse for handing in a late assignment, or
missing a lecture, please complete an M-Form and upload supporting
evidence at the following link,
\color{blue}\href{https://workflow.ul.ie/MForms/MForms/}{here}\color{black}.

\subsubsection{Repeat Assessment}\label{repeat-assessment}

The repeat assessment will take the same form as above (with an essay
instead of an exam). There will be two short tasks to complete and an
essay to be submitted together. Task 1 will be worth 25\%, Task 2, 30\%
and the Essay worth 45\%.

\subsubsection{Feedback}\label{feedback}

Written feedback will be provided for both Tasks 1 and 2. It is expected
that points of improvement identified in Task 1 will be implemented in
Task 2. Feedback on the Exam will be available on request.

\pagebreak

\subsection{Schedule}\label{schedule}

In Table 4 the schedule of topics to be covered in each lecture is
provided. Key readings for each topic are also identified. Further
direction towards additional readings regarding each topic will be
provided during lectures.

\begin{table}[h!]
  \begin{center}
    \begin{tabular}{| c | >{\arraybackslash}m{5cm} | >{\centering\arraybackslash}m{1.75cm} | >{\centering\arraybackslash}m{1.75cm} | >{\centering\arraybackslash}m{1.75cm} | >{\centering\arraybackslash}m{3cm} | }
    \multicolumn{6}{l}{\textbf{Table 4:} Schedule of lectures, topics and relevant readings} \\
    \hline
    & & \multicolumn{4}{c|}{\textbf{Readings}} \\ \hline
    \textbf{Week} & \textbf{Topic} &                      \textbf{Ranyard (2018)} & \textbf{Altman (2017)} & \textbf{Camerer et al. (2003)} & \textbf{Other}  \\ \hline
      1   & General Introduction and History of Economic Psychology & Chapter 1    & Chapter 1          & Chapter 1           &  \href{https://learn.ul.ie//content/enforced/46950-PS4168_SEM1_2024_5/1.Resources/Cartwright_2014_Chapter1.pdf}{\color{blue}{Cartwright (2014) Chapter 1}\color{black}} \\ \hline
      2   & Heuristics \& Biases                                    & \href{https://learn.ul.ie//content/enforced/46950-PS4168_SEM1_2024_5/1.Resources/Ranyard_2018_Chapter2.pdf}{\color{blue}{Chapter 2}\color{black}}    & Chapters 6 \& 7    &                     & Cartwright (2014) Chapter 2     \\ \hline
      3   & \makecell[l]{Nudges \\ (Task 1 announced) }           & (Chapter 23) & Chapter 17         &                     & \href{https://learn.ul.ie//content/enforced/46950-PS4168_SEM1_2024_5/1.Resources/Frantz_2016_Chapter8.pdf}{\color{blue}{Frantz et al., (2016) Chapter 8}\color{black}}; Thaler \& Sunstein (2008) \\ \hline
      4   & Theories of Decision Making                  & \href{https://learn.ul.ie//content/enforced/46950-PS4168_SEM1_2024_5/1.Resources/Ranyard_2018_Chapter2.pdf}{\color{blue}{Chapter 2}\color{black}}    & Chapters 2, 3 \& 4 & Chapters 2 \& 3     &                                 \\ \hline
      5   & Introduction to Game Theory            &              & \href{https://learn.ul.ie//content/enforced/46950-PS4168_SEM1_2024_5/1.Resources/Altman_2017_Chapter32.pdf}{\color{blue}{Chapter 32}\color{black}}         & Chapters 12 \& 13   & \href{https://learn.ul.ie//content/enforced/46950-PS4168_SEM1_2024_5/1.Resources/Antonides_1996_Chapter14.pdf}{\color{blue}{Antonides, (1996) Chapters 14}\color{black}}   \& 15 \\ \hline
      6  & \multicolumn{5}{c|}{\cellcolor{blue!25}\textbf{Task 1 Deadline:} Wednesday 16th October at 4pm (No Class: Open Days)} \\ \hline
      \multirow{3}{*}{7} & Emotional Influences          &              & \href{https://learn.ul.ie//content/enforced/46950-PS4168_SEM1_2024_5/1.Resources/Altman_2017_Chapter32.pdf}{\color{blue}{Chapter 32}\color{black}}         & Chapters 12 \& 13   & Antonides, (1996) Chapter 9 \\ \hhline{|~|~|-|-|-|-|}
         & \makecell[l]{Future Decisions \\ and Affective Forecasting \\ (Task 2 announced) }    & Chapter 3    & Chapters 23 \& 24  &
      \href{https://learn.ul.ie//content/enforced/46950-PS4168_SEM1_2024_5/1.Resources/Camerer_2003_Chapter6and7.pdf}{\color{blue}{Chapters 6 \& 7}\color{black}}     & Cartwright (2014) Chapter 4     \\ \hline
      8   & Loss Aversion/Risk Aversion and Endowment Effects       & \href{https://learn.ul.ie//content/enforced/46950-PS4168_SEM1_2024_5/1.Resources/Ranyard_2018_Chapter2.pdf}{\color{blue}{Chapter 2}\color{black}}    & Chapter 11         & Chapter 2, 4 \& 5   & Cartwright (2014) Chapter 3     \\ \hline
      9  & Mental Accounting                                       & Chapter 8    &                    & \href{https://learn.ul.ie//content/enforced/46950-PS4168_SEM1_2024_5/1.Resources/Camerer_2003_Chapter3.pdf}{\color{blue}{Chapter 3}\color{black}} \& 14     &                                 \\ \hline
      \multirow{2}{*}{10} & \multicolumn{5}{c|}{\cellcolor{blue!25}\textbf{Task 2 Deadline} (Wednesday 13th November at 4pm)} \\ \hhline{|~|-|-|-|-|-|}
        & Fairness and Ethics                                     &              & \href{https://learn.ul.ie//content/enforced/46950-PS4168_SEM1_2024_5/1.Resources/Altman_2017_Chapter32.pdf}{\color{blue}{Chapter 32}\color{black}}         & Chapters 8, 9 \& 18 &                                 \\ \hline
      11  & Gambling                                                & \href{https://learn.ul.ie//content/enforced/46950-PS4168_SEM1_2024_5/1.Resources/Ranyard_2018_Chapter19.pdf}{\color{blue}{Chapter 19}\color{black}}   &                    & Chapter 23          &                                 \\ \hline
      12  & Review                            &                    &                    &              &                                 \\ \hhline{|-|-|-|-|-|-|} 
      13  & \multicolumn{5}{c|}{Study Week} \\ \hline
      14/15  & \multicolumn{5}{c|}{\cellcolor{blue!25}\textbf{Exam} (Exam Week)} \\ \hline
      \end{tabular}
  \end{center}
\end{table}

\pagebreak

\subsection{Reading List}\label{reading-list}

Below is list of relevant readings. This list reflects the key topics to
be covered in this module. Additional readings may be identified during
lectures. Links to the library locations are provided for selected
books.

\subsubsection{Core Readings Materials}\label{core-readings-materials}

\noindent \vspace{-2em} \setlength{\parindent}{-0.5in}
\setlength{\leftskip}{0.5in} \setlength{\parskip}{7.5pt}

Altman, M. (2017).
\href{https://uol.primo.exlibrisgroup.com/discovery/fulldisplay?docid=alma991003444459703496&context=L&vid=353UOL_INST:353UOL_VU1&lang=en&search_scope=MyInst_and_CI&adaptor=Local\%20Search\%20Engine&tab=TAB1&query=any,contains,Handbook\%20of\%20behavioral\%20economics\%20and\%20smart\%20decision-making.&sortby=rank&facet=rtype,include,books&offset=0}{\color{blue}\emph{Handbook
of behavioral economics and smart decision-making.}\color{black}}
Cheltenham: Edward Elgar.

Camerer, C., Loewenstein, G., \& Rabin, M. (2003).
\href{https://uol.primo.exlibrisgroup.com/discovery/fulldisplay?docid=alma991001289209703496&context=L&vid=353UOL_INST:353UOL_VU1&lang=en&search_scope=MyInst_and_CI&adaptor=Local\%20Search\%20Engine&tab=TAB1&query=any,contains,Advances\%20in\%20behavioral\%20economics&sortby=rank&offset=0}{\color{blue}\emph{Advances
in behavioral economics.}\color{black}} Princeton, N.J.; Princeton
University Press.

Cartwright, E. (2014). \emph{Behavioral economics.} New York: Routledge.

Ranyard, R. (Ed.). (2018). \emph{Economic psychology.} Hoboken, NJ
Chichester, West Sussex: Wiley.

\setlength{\leftskip}{0in}

\subsubsection{Further Reading}\label{further-reading}

\noindent \vspace{-2em} \setlength{\parindent}{-0.5in}
\setlength{\leftskip}{0.5in} \setlength{\parskip}{7.5pt}

Antonides, G. (1996). \emph{Psychology in Economics and Business: An
Introduction to Economic Psychology.} Dordrecht: Kluwer.

Dowling, J. M. (2007).
\href{https://uol.primo.exlibrisgroup.com/discovery/fulldisplay?docid=alma991001565039703496&context=L&vid=353UOL_INST:353UOL_VU1&lang=en&search_scope=MyInst_and_CI&adaptor=Local\%20Search\%20Engine&tab=TAB1&query=any,contains,Modern\%20developments\%20in\%20behavioral\%20economics:\%20social\%20science\%20perspectives\%20on\%20choice\%20and\%20decision\%20making&sortby=rank&offset=0}{\color{blue}\emph{Modern
developments in behavioral economics: social science perspectives on
choice and decision making.}\color{black}} Singapore; World Scientific.

Frantz, R., Chen, S.-H., Dopfer, K., Heukelom, F., \& Mousavi, S.
(2016). \emph{Routledge Handbook of Behavioral Economics.} Routledge.

Gigerenzer, G. (2007).
\href{https://uol.primo.exlibrisgroup.com/discovery/fulldisplay?docid=alma991001060349703496&context=L&vid=353UOL_INST:353UOL_VU1&lang=en&search_scope=MyInst_and_CI&adaptor=Local\%20Search\%20Engine&tab=TAB1&query=any,contains,gut\%20feelings&sortby=rank}{\color{blue}\emph{Gut
feelings: short cuts to better decision making.}\color{black}} London:
Penguin.

Gigerenzer, G., \& Selten, R. (2002). \emph{Bounded Rationality: The
Adaptive Toolbox.} MIT Press.

Hogarth, R. M., \& Reder, M. W. (1987).
\href{https://uol.primo.exlibrisgroup.com/discovery/fulldisplay?docid=alma991003403189703496&context=L&vid=353UOL_INST:353UOL_VU1&lang=en&search_scope=MyInst_and_CI&adaptor=Local\%20Search\%20Engine&tab=TAB1&query=any,contains,Rational\%20Choice:\%20The\%20Contrast\%20between\%20Economics\%20and\%20Psychology&sortby=rank}{\color{blue}\emph{Rational
choice: the contrast between economics and psychology}\color{black}}.
Chicago: University of Chicago Press.

Kahneman, D. (2011).
\href{https://uol.primo.exlibrisgroup.com/discovery/fulldisplay?docid=alma991001624339703496&context=L&vid=353UOL_INST:353UOL_VU1&lang=en&search_scope=MyInst_and_CI&adaptor=Local\%20Search\%20Engine&tab=TAB1&query=any,contains,Thinking,\%20fast\%20and\%20slow.&sortby=rank&offset=0}{\color{blue}\emph{Thinking,
fast and slow.}\color{black}} London: Allen Lane.

Lewis, A. (2008).
\href{https://uol.primo.exlibrisgroup.com/discovery/fulldisplay?docid=alma991003792192903496&context=L&vid=353UOL_INST:353UOL_VU1&lang=en&search_scope=MyInst_and_CI&adaptor=Local\%20Search\%20Engine&tab=TAB1&query=any,contains,The\%20Cambridge\%20handbook\%20of\%20psychology\%20and\%20economic\%20behaviour.&sortby=rank&offset=0}{\color{blue}\emph{The
Cambridge handbook of psychology and economic behaviour.}\color{black}}
Cambridge: Cambridge University Press.

Nutt, P. C., \& Wilson, D. C. (2010).
\href{https://uol.primo.exlibrisgroup.com/discovery/fulldisplay?docid=alma991001512319703496&context=L&vid=353UOL_INST:353UOL_VU1&lang=en&search_scope=MyInst_and_CI&adaptor=Local\%20Search\%20Engine&tab=TAB1&query=any,contains,Handbook\%20of\%20decision\%20making.&sortby=rank&offset=0}{\color{blue}\emph{Handbook
of decision making.}\color{black}} Oxford: Wiley-Blackwell.

Thaler, R. H. (2015).
\href{https://uol.primo.exlibrisgroup.com/discovery/fulldisplay?docid=alma991003236569703496&context=L&vid=353UOL_INST:353UOL_VU1&lang=en&search_scope=MyInst_and_CI&adaptor=Local\%20Search\%20Engine&tab=TAB1&query=any,contains,Misbehaving:\%20The\%20Making\%20of\%20Behavioral\%20Economics.&sortby=rank&offset=0}{\color{blue}\emph{Misbehaving:
how economics became behavioural.\color{black}}} W.W. Norton.

\setlength{\leftskip}{0in}

\subsubsection{Additional Resources}\label{additional-resources}

\noindent \vspace{-2em} \setlength{\parindent}{-0.5in}
\setlength{\leftskip}{0.5in} \setlength{\parskip}{7.5pt}

Axelrod, R., \& Hamilton, W. D. (1981). The evolution of cooperation.
\emph{Science}, 211(4489), 1390--1396. \color{blue}
\url{https://doi.org/10.1126/science.7466396} \color{black}

Camerer, C. (1999). Behavioral economics: Reunifying psychology and
economics. \emph{Proceedings of the National Academy of Sciences},
96(19), 10575--10577. \color{blue}
\url{https://doi.org/10.1073/pnas.96.19.10575} \color{black}

Critchfield, T. S., \& Kollins, S. H. (2001). Temporal Discounting:
Basic Research and the Analysis of Socially Important Behavior.
\emph{Journal of Applied Behavior Analysis}, 34(1), 101--122.
\color{blue} \url{https://doi.org/10.1901/jaba.2001.34-101}
\color{black}

Davis, M. D. (2012). \emph{Game Theory: A Nontechnical Introduction}.
Courier Corporation.

Frey, B. S., \& Stutzer, A. (Eds.). (2007). \emph{Economics and
Psychology: A Promising New Cross-Disciplinary Field}. Cambridge, Mass:
The MIT Press.

Kahneman, D., \& Tversky, A. (Eds.). (2000). \emph{Choices, Values, and
Frames} (1 edition). New York: Cambridge, UK: Cambridge University
Press.

Kruglanski, A. W., \& Gigerenzer, G. (2011). Intuitive and deliberate
judgments are based on common principles. \emph{Psychological Review},
118(1), 97--109. \color{blue} \url{https://doi.org/10.1037/a0020762}
\color{black}

Lea, S. E. G., Tarpy, R. M., \& Webley, P. M. (1987). \emph{The
Individual in the Economy: A Textbook of Economic Psychology}. Cambridge
Cambridgeshire; New York: Cambridge University Press.

Lea, S. E. G., Webley, P., \& Young, B. M. (1992). \emph{New Directions
in Economic Psychology}. Edward Elgar Publishing.

Raaij, W. F., Veldhoven, G. M., \& Wärneryd, K.-E. (1988).
\emph{Handbook of Economic Psychology}. Dordrecht: Springer Netherlands.
\color{blue} \url{http://dx.doi.org/10.1007/978-94-015-7791-5}
\color{black}

Thaler, R. H., \& Sunstein, C. R. (2008). \emph{Nudge: improving
decisions about health, wealth, and happiness}. New Haven: Yale
University Press.

Todd, P. M., \& Gigerenzer, G. (Eds.). (2012). \emph{Ecological
rationality: intelligence in the world}. Oxford; New York: Oxford
University Press.

\end{document}
